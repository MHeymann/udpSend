\documentclass[a4paper,10pt]{article}
\usepackage{graphicx}
\usepackage[hidelinks]{hyperref}

\title{CS 354\\Tutorial 2: RBUDP}
\author{M. Heymann\\15988694@sun.ac.za \and J. Lombard\\16994914@sun.ac.za}
\date{16 August 2017}

\begin{document}

\clearpage\maketitle
\thispagestyle{empty} % make sure no page numbers on title page

\pagebreak

\clearpage\tableofcontents
\thispagestyle{empty}

\pagebreak
\setcounter{page}{1} % Set the page counter to 1

\section{Introduction}
The purpose of this project is to understand UDP and TCP for transferring files
over a network.\\\\
Reliable Blast User Datagram Protocol is a data transfer tool and protocol
specifically designed to move very large files over wide area high-speed
networks \cite{udpDef}.\\\\
Transmission Control Protocol/Internet Protocol is the communication language
used for the Internet and it can also be used on private networks
\cite{tcpDef}.\\
The protocols map to a four-layer model \cite{map}:
\begin{enumerate}
	\item Network Interface (Ethernet)
	\item Network (IP)
	\item Transport (TCP, UDP)
	\item Application (HTTP)
\end{enumerate}
The reliability and transfer rates of these protocols are compared.

\section{Unimplemented features}
All the specified features were implemented.

\section{Description of files}
The program consists of 12 Java files.

\subsection{ShareFile.java}
ShareFile.java is the driver file for UDP transfers.

\subsection{ShareTCP.java}
ShareTCP.java is the driver file for TCP transfers.

\subsection{filereceiver}
\subsubsection{Receiver.java}
Receiver.java creates the GUI that enables receiving files with UDP.
\subsubsection{ReceiverReconstructor.java}
ReceiverReconstructor.java runs on a separate thread. It receives packets
individually and order the packets by sequence number.
The packets are then written to a new file, of which the file name was
specified by the receiving user when transfer begins.

\subsection{filesender}
\subsubsection{Sender.java}
Sender.java creates the GUI that enables sending files with UDP.
\subsubsection{SenderDeconstructor.java}
SenderDeconstructor.java runs on a separate thread. The files are broken up
into datagram packets with unique sequence numbers. The packets are then sent
using UDP.

\subsection{layouts}
\subsubsection{RelativeLayout.java}
RelativeLayout.java is an external library for the general layout of the GUIs.

\subsection{packet}
\subsubsection{Packet.java}
Packet.java is a datagram packet abstraction.
\begin{itemize}
	\item int sequenceNumber - orders the packets upon reconstruction of
		the file
	\item int size - The total size of the packet payload.
	\item byte[] data - The data being sent in this packet.
\end{itemize}

\subsection{parameters}
\subsubsection{Parameters.java}
Parameters.java contains global parameters that need to be consistent
throughout the source files.

\subsection{tcpreceiver}
\subsubsection{Receiver.java}
Receiver.java creates the GUI that enables receiving files with TCP.
\subsubsection{ReceiverReconstructor.java}
ReceiverReconstructor.java runs on a separate threads. It receives packets
individually and writes the packets to a file.

\subsection{tcpsender}
\subsubsection{Sender.java}
Sender.java creates the GUI that enables sending files with TCP.
\subsubsection{SenderDeconstructor.java}
SenderDeconstructor.java runs on a separate thread. The files are broken up
into packets which are sent using TCP.

\section{Program description}
Initially the sender and receiver launch their GUI clients on their
respective computers.  The sender sets receiver's IP, selects a file and
hits send.

Before anything else, the number of datagram packets that the file consists
of gets sent to the receiver.  This is mostly to enable percentage tracking
of the file as it gets sent.  

Batches of datagrams are sent, at the end of which the sender requests
acknowledgements of each. This acknowledgement happens over TCP. If a
packet didn't make it to the receiver, this is when the sender gets
notified. Only when all packets in the batch was acknowledged does
transmission continue with the next batch. Also, once a batch has been
completely acknowledged, the receiver writes the whole batch to file. 

At the end of transmission, and end of file signal gets sent to the
receiver, who then writes the final packets to the file.  

\section{Experiments}
\subsection{Throughput of TCP and RBUDP data transfer}

Different file sizes were sent over the connection, up to a size of 200 MB.
the time taken by TCP and RUDP respectively was noted. TCP is clearly
more efficient than this implementation of RUDP.
\begin{tabular}{|ccc|r@.l|r@.l|}
	\hline
	\multicolumn{3}{|c|}{File size} & \multicolumn{4}{|c|}{Time (seconds)}\\
	\hline
	\multicolumn{3}{|c|}{(bytes)} & \multicolumn{2}{c|}{TCP} &
	\multicolumn{2}{c|}{UDP}\\
	\hline
	& 264 & 228 & 0 & 01 & 0 & 01\\
	\hline
	& 368 & 618 & 0 & 01 & 0 & 01\\
	\hline
	3 & 029 & 804 & 0 & 01 & 0 & 01\\
	\hline
	89 & 740 & 009 & 0 & 01 & 0 & 91\\
	\hline
	102 & 802 & 047 & 0 & 01 & 0 & 91\\
	\hline
	140 & 062 & 720 & 0 & 01 & 1 & 31\\
	\hline
	199 & 273 & 479 & 0 & 11 & 1 & 91\\
	\hline
\end{tabular}

\subsection{Transfer rate of RBUDP}
\begin{tabular}{|cccc|r@.l|}
	\hline
	\multicolumn{4}{|c|}{File size} & \multicolumn{2}{|c|}{Time}\\
	\hline
	\multicolumn{4}{|c|}{(bytes)} &
	\multicolumn{2}{c|}{(seconds)}\\
	\hline
	& & 264 & 228 & 0 & 01\\
	\hline
	& & 368 & 618 & 0 & 01\\
	\hline
	& 3 & 029 & 804 & 0 & 01\\
	\hline
	& 89 & 740 & 009 & 0 & 91\\
	\hline
	& 102 & 802 & 047 & 0 & 91\\
	\hline
	& 140 & 062 & 720 & 1 & 31\\
	\hline
	& 199 & 273 & 479 & 1 & 91\\
	\hline
	& 870 & 701 & 581 & 10 & 81\\
	\hline
	1 & 235 & 459 & 142 & 17 & 11\\
	\hline
	1 & 900 & 032 & 010 & 35 & 51\\
	\hline
	2 & 060 & 362 & 926 & 47 & 01\\
	\hline
\end{tabular}

\subsection{Different packet sizes in RBUDP}
Different payload sizes for UDP datagrams was used and the transfer time
was noted, for various file sizes.  Having a datagram payload size of about
50 KB did speed up transfer rates marginally.
\\
\begin{tabular}{|cccc|r@.l|r@.l|r@.l|}
	\hline
	\multicolumn{4}{|c|}{File size} & \multicolumn{6}{c|}{Time (seconds)}\\
	\hline
	\multicolumn{4}{|c|}{(kilobytes)} &
	\multicolumn{6}{c|}{Packet size (bytes)}\\
	\hline
	& & & & \multicolumn{2}{c|}{5} & \multicolumn{2}{c|}{25} &
	\multicolumn{2}{c|}{50}\\
	\hline
	& & 264 & 228 & 0 & 01 & 0 & 01 & 0 & 01\\
	\hline
	& & 368 & 618 & 0 & 01 & 0 & 01 & 0 & 01\\
	\hline
	& 3 & 029 & 804 & 0 & 01 & 0 & 01 & 0 & 01\\
	\hline
	& 89 & 740 & 009 & 0 & 91 & 0 & 81 & 0 & 41\\
	\hline
	& 102 & 802 & 047 & 0 & 91 & 0 & 91 & 0 & 31\\
	\hline
	& 140 & 062 & 720 & 1 & 31 & 0 & 41 & 0 & 31\\
	\hline
	& 199 & 273 & 479 & 1 & 91 & 1 & 41 & 0 & 61\\
	\hline
	& 870 & 701 & 581 & 10 & 81 & 11 & 71 & 8 & 41\\
	\hline
	1 & 235 & 459 & 142 & 17 & 11 & 3 & 71 & 22 & 61\\
	\hline
	1 & 900 & 032 & 010 & 35 & 51 & 31 & 11 & 31 & 91\\
	\hline
	2 & 060 & 362 & 926 & 47 & 01 & 48 & 71 & 40 & 81\\
	\hline
\end{tabular}

\subsection{Varying packet loss rates}
The sender was artificially altered to drop one packet in every 128 burst.
this slowed the transfer rate for a 701 MB file from 72.81s to 74.81s.

\section{Issues encountered}
An issue encountered while writing this program was having the UDP socket set
as blocking caused gridlocks in the program when sending large files.
Although this is a minor bug, it was difficult to pinpoint.\\
Another minor issue was when files larger than 2.2GB is sent using UDP. The integer
used for percentage tracking wraps around and the values becomes
meaningless.

\section{Significant data structures}
A new defined datastructure for the program is a Packet with the fields:
\begin{itemize}
	\item int sequenceNumber
	\item int size
	\item byte[] data
\end{itemize}
Another data structure used in this program is a priority queue of
packets. The packets are ordered in the queue by sequence number.

\section{Design}
For this program it was decided to used fixed burst sizes instead of a moving
send window. Before the next burst is sent an acknowledgement of the previous
burst must be received.\\
Another decision made was to use the Java NIO datagram channel instead of
the older Java IO.

\section{Compilation and Execution}
The program contains make and run files.\\\\
To compile the RUDP application:
\begin{verbatim}
	cd src
	./make.sh
\end{verbatim}
To compile the TCP application:
\begin{verbatim}
	cd src
	./maketcp.sh
\end{verbatim}
To run the UDP sender:
\begin{verbatim}
	./run.sh send
\end{verbatim}
To run the UDP receiver:
\begin{verbatim}
	./run.sh receive
\end{verbatim}
To run the TCP sender:
\begin{verbatim}
	./runtcp.sh send
\end{verbatim}
To run the TCP receiver:
\begin{verbatim}
	./runtcp.sh receive
\end{verbatim}

\section{Conclusion}
From the experiments it is clear that TCP is faster several by orders of magnitude.
Although our RUDP application is reliable it seems pure TCP is still more
efficient.\\\\
Experimenting with packet sizes shows that larger packets may not necessarily
speed up the transfer significantly.
Experimenting with packet loss shows that losing packets slows the transfer
rate noticeably.\\\\
The experiments also revealed that the integer used for percentage tracking
should have been a long int.

\pagebreak
\bibliographystyle{plain}
\bibliography{references.bib}

\end{document}
